\chapter{Fishing for Binaries}

\section{Binary Dynamics}

The next day, our three friends get together again.  Alice looks a bit
sleepy, obviously has stayed up late the previous night.

\abc

\carol
Fortunately, we have a blackboard in this room.

\bob
And more important, we have Alice in the room!

\alice
But only a half-awake Alice.  Yesterday evening it took me longer than
I thought to figure everything out.  What I found looks easy, it always
does in the end, but I didn't realize how much I had forgotten about
the classical dynamics course that I had followed in my freshman year.

\bob
Some day I'd like to know how you derived whatever you derived, but
for now, I'm really eager to fish for binaries, so let's not worry too
much about derivations.  Perhaps you can just show us what you found.

\carol
I second that.  Let's see whether it works first, and if it turns out
to be useful, I'd love to learn more about the details.

\alice
Fine with me, and just as well, since I don't think I'm quite together
enough to give a complete pedagogical derivation.

\cba

Alice walks to the blackboard, and writes down the equations for the
energy $E$ and angular momentum $L$ for a two-body system.  Here the
masses of the two bodies are $M_1$ and $M_2$.  The relative position
vector, pointing from $M_1$ to $M_2$, is given by $\br$, and its time
derivative is the relative velocity vector $\bv$.  As before, $r$ is
the absolute value of $\br$, a scalar, equal to the length of the
vector $\br$, and similar $v$ is the length of $\bv$.  The energy is
given by the sum of the negative potential energy and the positive
kinetic energy.  The angular momentum points along the outer product,
also called the cross product, of the relative position and velocity
vectors.  This implies that it points in a direction perpendicular to
the orbit of the two bodies.

\begin{equation}
E = -\, G\:\frac{M_1 M_2}{r} \;+\, \half\,\frac{M_1 M_2}{M_1 + M_2}\:v^2
\end{equation}

\begin{equation}
L = \frac{M_1 M_2}{M_1 + M_2} \; \br \times \bv
\end{equation}

\noindent
Note that we could have defined $\br$ equally well as pointing from
$M_2$ to $M_1$, as long as $\bv$ would still be the time derivative of
$\br$.  In the definition of the energy, the lengths of $r$ and $v$
would not change, and in the definition of the angular momentum, the
two minus signs that would thus be introduced at the right would
cancel each other.

These equations simplify considerably when we introducing the symbol
$\mu$ for the reduced mass, defined as follows:

\begin{equation}
\mu = \frac{M_1 M_2}{M_1 + M_2}
\end{equation}

\noindent
we can now write the energy per unit of reduced mass as

\begin{equation}
\tilde E  \equiv \frac{E}{\mu} = -\, G\:\frac{M_1 + M_2}{r} \;+\, \half\,\:v^2
\end{equation}

\noindent
and similarly the angular momentum per unit of reduced mass as

\begin{equation}
\tilde L \equiv \frac{L}{\mu}  =  \br \times \bv
\end{equation}

\abc

\alice
The shape of a binary orbit is given by the values of the semi-major
axis $a$ and the eccentricity $e$.  We can find those values in two steps.
First we invert the expression that gives $\tilde E$ in terms of $a$,
to obtain an expression for $a$:

\cba

\begin{equation}
\tilde E = -\, G\:\frac{M_1 + M_2}{2 a}
\end{equation}

\begin{equation}
a = -\, G\:\frac{M_1 + M_2}{2 \tilde E}\label{eq:a}
\end{equation}

\abc

\alice
And similarly, we invert the expression that gives $\tilde L$ in terms
of $a$ and $e$, to obtain an expression for $e$, or to keep it simple,
for $e^2$ (we can always take the square root later):

\cba

\begin{equation}
\tilde L^2 = G\: (M_1 + M_2)a(1-e^2)
\end{equation}

\begin{equation}
e^2 = 1 - \frac{\tilde L^2}{G(M_1 + M_2)a}\label{eq:e}
\end{equation}

\abc

\alice
The interpretation is as follows: as the name suggests, $a$ is half
the length of the longest axis of the ellipse that describes the
relative orbit of the two bodies; in this way $2a$ is size of the orbit.
The eccentricity $e$ indicates the relative displacement of the focus
of the orbit from the center of the orbit.  The closest approach
between the two bodies occurs at a distance of $ae$, also called the
pericenter distance, while the largest separation occurs at a distance
of $a(1-e)$, the apocenter distance.

\cba

\section{Finding Binaries}

\abc

\carol
Thanks, Alice, let's code it up.

\cba

While Bob and Alice look on, and occasionally give some comments,
Carol produces the following code.  First, she summarized the
structure and usage:

\code{find\_binaries1.C: summary}{chap11/find_binaries1.C.1_summary}

followed by the usual include stuff and declarations, etc.:

\code{find\_binaries1.C: premain}{chap11/find_binaries1.C.2_premain}

The main part of the code is rather simple: one by one the snapshots
are read in and a search for binaries is done, by invoking the
function {\st find\_binaries}:

\code{find\_binaries1.C: main}{chap11/find_binaries1.C.3_main}

There are no command line arguments, apart from the standard help option:

\code{find\_binaries1.C: read\_options}{chap11/find_binaries1.C.read_options}

We have seen before how to read in a snapshot:

\code{find\_binaries1.C: get\_snapshot}{chap11/find_binaries1.C.get_snapshot}

Also, we have seen before how to delete in a snapshot once it is no
longer needed:

\code{find\_binaries1.C: delete\_snapshot}
     {chap11/find_binaries1.C.delete_snapshot}

Finally, here is the code where the real work is done.  The equations
written above are used directly to find and report the values for the
semi-major axis $a$ and eccentricity $e$.  Each particle pair that has
a negative relative energy is considered to be a bound system, in
other words a binary:

\code{find\_binaries1.C: find\_binaries}{chap11/find_binaries1.C.find_binaries}

\abc

\bob
I'm really curious now what will happen.  How about starting
carefully, in a number of easy steps, since we're entering new and
unknown terrain here.

\carol
Now you begin to talk like a computer scientist!  But I agree, let's
feel our way around, to make sure we understand what we're doing.

\alice
The obvious first step would be to look at the output of {\st sphere1.C},
to see whether there would be any binaries already present.

\bob
Let's choose a particle seed, so that we can reproduce our results.

\cba

\begin{small}
\begin{verbatim}
|gravity> ../chap9/sphere1 -s 42 -n 25 | find_binaries1
seed = 42
time = 0
star1 = 0  star2 = 1  a = 0.856428  e = 1
star1 = 0  star2 = 2  a = 0.397074  e = 1
star1 = 0  star2 = 3  a = 0.331345  e = 1
 . . . .
star1 = 22  star2 = 23  a = 0.683746  e = 1
star1 = 22  star2 = 24  a = 0.562686  e = 1
star1 = 23  star2 = 24  a = 0.330094  e = 1
|gravity> 
\end{verbatim}
\end{small}

\abc

\carol
Wow, that went by too quick, but there sure were a lot of so-called
binaries.  How many were there altogether?

\bob
Let see.  Here is the answer:

\cba

\begin{small}
\begin{verbatim}
|gravity> ../chap9/sphere1 -s 42 -n 25 | find_binaries1 | wc
seed = 42
    301    3603   12941
|gravity> 
\end{verbatim}
\end{small}

\abc

\bob
We know that there is a line {\st time = 0}, so if we leave that line
out, we realize that we have just found a grand total of 300 binaries.
And all that with only 25 stars ?!?

\alice
I see what the problem is.  We begin with a cold start, remember?  All
velocities are set equal to zero, initially.

\carol
So this means the kinetic energy of any and every pair of stars is
identically zero.

\bob
And therefore the total energy is negative, since potential energy is
always negative.  So we should have found $25 * 25 = 625$ binaries, no?

\alice
You can't be a binary with yourself, so there are only $25 * 24$
candidates.  However, that still leaves us with 600 binaries.

\carol
Ah, but remember, I tested each pair only once, independent of whether
I started with the first or the second star.  In other words, for the
first star, I did 24 tests with all other stars; for the second star I
needed only to test binarity with respect to the other 23 stars; and
so on, until the next-to-last star only had to be tested with respect
to the last star, and we were done.  This means that there are only
$(25*24)/2 = 300$ star pairs that were tested.

\bob
Indeed just what we found.  Phew, I'm relieved.  At least the output
makes sense, even this is not quite what we intended.

\alice
Notice the typical semi-major axis values in our output: all these
binary-wanna-be systems claim to have $a$ values spread more or less
equally between 0 and 1.  In practice, any stable binary would have an
orbit far smaller than the size of the system.

\carol
How about adding an option to the code through which we exclude all
binaries that are larger than a certain cut-off value?

\bob
That should help, yes, let's try that.

\cba

\section{Finding Tight Binaries}

Bob and Alice again watch Carol, who quickly copies the file
{\st find\_binaries1.C} to a new file {\st find\_binaries2.C},
and then starts editing the new file, in order to add a cut-off
option.

\abc

\carol
Here is the new main part.  As you can see, I have added the cut-off
value {\st a\_max} for the semi-major axis of the binaries to be
reported as an optional argument.  Running {\st find\_binaries2.C} in
the same way as {\st find\_binaries1.C} should give the same output,
since the default cutoff is a {\st VERY\_LARGE\_NUMBER}, in our case
equal to $10^{300}$, which means that no binary found will be rejected.

\cba

\code{find\_binaries2.C: main}{chap11/find_binaries2.C.3_main}

Here the option is added to the parsing of the command line argument:

\code{find\_binaries2.C: read\_options}{chap11/find_binaries2.C.read_options}

And here the extra test is implemented in the actual search routine.

\code{find\_binaries2.C: find\_binaries}{chap11/find_binaries2.C.find_binaries}

\abc

\bob
Let's try to see first whether your claim is correct, that 
{\st find\_binaries2.C} gives the same output as {\st
find\_binaries1.C}.

\carol
Okay, here is the word count:

\cba

\begin{small}
\begin{verbatim}
|gravity> ../chap9/sphere1 -s 42 -n 25 | find_binaries2 | wc
seed = 42
    301    3603   12941
|gravity> 
\end{verbatim}
\end{small}

\abc

\carol
And to make really sure, here is a diff between the two outputs:

\cba

\begin{small}
\begin{verbatim}
|gravity> ../chap9/sphere1 -s 42 -n 25 | find_binaries1 > tmp1
seed = 42
|gravity> ../chap9/sphere1 -s 42 -n 25 | find_binaries2 > tmp2
seed = 42
|gravity> diff tmp1 tmp2
|gravity> 
\end{verbatim}
\end{small}

\abc

\bob
I'm convinced!  Now what shall we choose for {\st a\_max}?  How about
$a_{max} = 0.1$?

\carol
Fine, but let's start with a word count.

\cba

\begin{small}
\begin{verbatim}
|gravity> ../chap9/sphere1 -s 42 -n 25 | find_binaries2 -a 0.1 | wc
seed = 42
      4      39     142
|gravity> 
\end{verbatim}
\end{small}

\abc

\bob
A lot better already.  Discounting again the line {\st time = 0}, we
have found 3 tight binaries.

\alice
But even those binaries are likely to be not physical, but just chance
positionings of two particles that happen to be within a distance of
$0.1$ of each other, in our units.  Since the velocities are zero, any
particle pair that close will be counted.  We have only shown that 3
out of 300 pairs are this close, just 1\% of the possible pair choices.

\bob
I would have expected something like 0.1\%, one in a thousand, since
the room each particle has for harboring a neighbor at such a small
distance is the volume of a sphere with radius $0.1$.  And in three
dimensions that volume is a thousand times smaller than the volume of
the sphere in which particles are located, which has radius unity.

\carol
Ah, but there are 25 particles that you can start with; what you just
derived was the chance to find a binary companion around a particular
star.

\alice
And then we have to divide by two to avoid each pair from being
counted twice.  So we wind up with an estimate of 1.25\%, so we could
have expected $0.0125*300=4$ stars, on average.  To find 3 seems close
enough.

\bob
Here are a few more, now for random seeds

\cba

\begin{small}
\begin{verbatim}
|gravity> ../chap9/sphere1 -n 25 | find_binaries2 -a 0.1 | wc
seed = 1064763074
      5      51     184
|gravity>  !!
../chap9/sphere1 -n 25 | find_binaries2 -a 0.1 | wc
seed = 1064763076
      4      39     142
|gravity>  !!
../chap9/sphere1 -n 25 | find_binaries2 -a 0.1 | wc
seed = 1064763078
      3      27      98
|gravity>  !!
../chap9/sphere1 -n 25 | find_binaries2 -a 0.1 | wc
seed = 1064763082
      3      27      97
|gravity>  !!
../chap9/sphere1 -n 25 | find_binaries2 -a 0.1 | wc
seed = 1064763084
      6      63     230
|gravity>  !!
../chap9/sphere1 -n 25 | find_binaries2 -a 0.1 | wc
seed = 1064763087
      3      27      98
|gravity>  !!
../chap9/sphere1 -n 25 | find_binaries2 -a 0.1 | wc
seed = 1064763090
      5      51     187
|gravity>  !!
../chap9/sphere1 -n 25 | find_binaries2 -a 0.1 | wc
seed = 1064763093
      4      39     140
|gravity>  !!
../chap9/sphere1 -n 25 | find_binaries2 -a 0.1 | wc
seed = 1064763097
      3      27      99
|gravity>  !!
../chap9/sphere1 -n 25 | find_binaries2 -a 0.1 | wc
seed = 1064763098
      4      39     142
|gravity>  
|gravity> 
\end{verbatim}
\end{small}

\abc

\bob
The average is still 3, not 4.  In any case, statistics is a
very tricky subject, and I wouldn't be surprised if the true value
would be slightly different from our rough estimate of 4.  For example,
there may be edge effects involved, for particles near the surface of
the sphere.  But I would be uncomfortable if the outcome would have
been vastly different from 4 on average.

\alice
Now that we're happy with the statistics, can you show the actual output?

\bob
Ah, of course, here it is:

\cba

\begin{small}
\begin{verbatim}
|gravity> ../chap9/sphere1 -s 42 -n 25 | find_binaries2 -a 0.1
seed = 42
time = 0
star1 = 6  star2 = 12  a = 0.0271531  e = 1
star1 = 8  star2 = 23  a = 0.0783721  e = 1
star1 = 13  star2 = 14  a = 0.0386034  e = 1
|gravity> 
\end{verbatim}
\end{small}

\abc

\bob
Looks pretty random to me.  And the eccentricities are all unity, as
I should have predicted, had I thought about it, given that the
velocities are zero.  Particles falling toward each other move on a
straight line, the extreme case of an ellipse with eccentricity 1.

\cba

\section{Dynamically Produced Binaries}

\abc

\carol
Why don't we do a real run now.

\bob
I'll use our workhorse, {\st nbody\_sh1.C} with options {\st -i} to include
output for time 0 as well, and {\st -e 10} to allow only energy diagnostics
at the beginning and at the end so as not to clutter the screen.

\cba

\begin{small}
\begin{verbatim}
|gravity> ../chap9/sphere1 -s 42 -n 25 | ../chap8/nbody_sh1 -e 10 -i | find_binaries2 -a 0.1
seed = 42
Starting a Hermite integration for a 25-body system,
  from time t = 0 with time step control parameter dt_param = 0.03  until time 10 ,
  with diagnostics output interval dt_dia = 10,
  and snapshot output interval dt_out = 1.
at time t = 0 , after 0 steps :
  E_kin = 0 , E_pot = -0.61885 , E_tot = -0.61885
                absolute energy error: E_tot - E_init = 0
                relative energy error: (E_tot - E_init) / E_init = -0
time = 0
star1 = 6  star2 = 12  a = 0.0271531  e = 1
star1 = 8  star2 = 23  a = 0.0783721  e = 1
star1 = 13  star2 = 14  a = 0.0386034  e = 1
time = 1.00008
star1 = 16  star2 = 21  a = 0.0601398  e = 0.612398
time = 2.00009
star1 = 1  star2 = 3  a = 0.0816514  e = 0.823595
star1 = 1  star2 = 4  a = 0.0200926  e = 0.945035
star1 = 5  star2 = 8  a = 0.0336744  e = 0.798637
star1 = 8  star2 = 23  a = 0.00884901  e = 0.744231
time = 3.00005
star1 = 0  star2 = 1  a = 0.048932  e = 0.926806
star1 = 0  star2 = 21  a = 0.0669483  e = 0.681452
star1 = 1  star2 = 3  a = 0.0668783  e = 0.994049
star1 = 1  star2 = 21  a = 0.041487  e = 0.87713
star1 = 5  star2 = 8  a = 0.0541946  e = 0.841467
star1 = 5  star2 = 23  a = 0.00772702  e = 0.905568
time = 4.00008
star1 = 0  star2 = 3  a = 0.080123  e = 0.872677
star1 = 0  star2 = 21  a = 0.0268892  e = 0.940599
star1 = 2  star2 = 11  a = 0.0061008  e = 0.807254
star1 = 2  star2 = 19  a = 0.0910958  e = 0.653393
time = 5.00001
star1 = 2  star2 = 19  a = 0.0226411  e = 0.927754
star1 = 3  star2 = 21  a = 0.0149472  e = 0.595139
star1 = 5  star2 = 17  a = 0.00895071  e = 0.422472
time = 6.00001
star1 = 2  star2 = 19  a = 0.0023379  e = 0.508743
star1 = 3  star2 = 21  a = 0.0159819  e = 0.556314
time = 7.00001
star1 = 0  star2 = 21  a = 0.0180656  e = 0.303921
star1 = 2  star2 = 19  a = 0.00233757  e = 0.507285
time = 8.00002
star1 = 0  star2 = 4  a = 0.0258779  e = 0.996719
star1 = 0  star2 = 21  a = 0.0279722  e = 0.883476
star1 = 2  star2 = 19  a = 0.00233754  e = 0.507268
star1 = 4  star2 = 21  a = 0.0805251  e = 0.390524
time = 9.00001
star1 = 0  star2 = 21  a = 0.0132626  e = 0.650153
star1 = 2  star2 = 19  a = 0.0023375  e = 0.506389
at time t = 10 , after 498929 steps :
  E_kin = 0.672558 , E_pot = -1.1658 , E_tot = -0.493243
                absolute energy error: E_tot - E_init = 0.125607
                relative energy error: (E_tot - E_init) / E_init = -0.202968
time = 10
star1 = 0  star2 = 3  a = 0.0243207  e = 0.965155
star1 = 0  star2 = 21  a = 0.0311286  e = 0.764542
star1 = 2  star2 = 19  a = 0.00233747  e = 0.506845
star1 = 3  star2 = 21  a = 0.0458811  e = 0.746161
|gravity> 
\end{verbatim}
\end{small}

\abc

\bob
I'm glad to see that only at time 0 we have those skinny binaries with
eccentricity 0.  And hey, look at that very tight binary, with a
semi-major axis of only $0.002$, formed by stars 2 and 19!

\carol
It is clearly a persistent binary.  It was already there at time 6,
with pretty much the same short semi-major axis.

\alice
Once a binary is so tight, it is less likely that any other star will
come close to that close pair, so it will take a long time before
there is another strong interaction.

\bob
I see what you mean: the same binary was present at time 5, but with a
much wider orbit, about ten times as large; and it did not last long
at that larger distance, before it got a lot tighter.

\alice
That event must have been a more complex three-body dance.  You can
see that at time 4, star 2 was simultaneously considered to be bound
to start 19 and to star 11.  It was far closer to start 11 at that
time, so the large binding energy must have come at least partly from
that pair.

\bob
Ho!  Before we analyze much further, I just noticed something really bad.
In our excitement about binaries, we forgot to check whether the energy
conservation was reasonable.  Look!  We have a relative energy error
of more than 20\%.

\carol
That's no good.  We should rerun with a smaller accuracy parameter for
the integrator.

\alice
I agree.  Still, the main points we made are likely to remain valid
qualitatively, even if not quantitatively.

\bob
I'll use the same seed again, but with a three times smaller step size
control parameter.  With a fourth-order integration scheme, that
should reduce the relative energy error for the whole system to much
less than 1\%.

\cba

\begin{small}
\begin{verbatim}
|gravity> ../chap9/sphere1 -s 42 -n 25 | ../chap8/nbody_sh1 -d 0.01 -e 10 -i | find_binaries2 -a 0.1
seed = 42
Starting a Hermite integration for a 25-body system,
  from time t = 0 with time step control parameter dt_param = 0.01  until time 10 ,
  with diagnostics output interval dt_dia = 10,
  and snapshot output interval dt_out = 1.
at time t = 0 , after 0 steps :
  E_kin = 0 , E_pot = -0.61885 , E_tot = -0.61885
                absolute energy error: E_tot - E_init = 0
                relative energy error: (E_tot - E_init) / E_init = -0
time = 0
star1 = 6  star2 = 12  a = 0.0271531  e = 1
star1 = 8  star2 = 23  a = 0.0783721  e = 1
star1 = 13  star2 = 14  a = 0.0386034  e = 1
time = 1.00008
star1 = 5  star2 = 18  a = 0.0804608  e = 0.373533
star1 = 7  star2 = 16  a = 0.0653846  e = 0.980501
star1 = 14  star2 = 16  a = 0.0592383  e = 0.88506
time = 2.00006
star1 = 7  star2 = 15  a = 0.0551175  e = 0.78881
star1 = 8  star2 = 10  a = 0.082922  e = 0.406961
time = 3.00003
star1 = 8  star2 = 13  a = 0.00555818  e = 0.849134
star1 = 10  star2 = 16  a = 0.0420807  e = 0.772984
star1 = 10  star2 = 19  a = 0.0345081  e = 0.967946
time = 4
star1 = 3  star2 = 10  a = 0.0645721  e = 0.651257
star1 = 7  star2 = 23  a = 0.0778391  e = 0.475478
star1 = 8  star2 = 13  a = 0.00554977  e = 0.677847
time = 5
star1 = 3  star2 = 8  a = 0.002126  e = 0.920892
star1 = 4  star2 = 22  a = 0.0423633  e = 0.638268
time = 6
star1 = 3  star2 = 8  a = 0.00212601  e = 0.952222
star1 = 4  star2 = 22  a = 0.0403322  e = 0.827227
time = 7
star1 = 3  star2 = 8  a = 0.00220527  e = 0.777804
time = 8
star1 = 3  star2 = 8  a = 0.00220456  e = 0.732745
time = 9
star1 = 3  star2 = 8  a = 0.00197343  e = 0.801175
at time t = 10 , after 2642122 steps :
  E_kin = 0.783526 , E_pot = -0.871684 , E_tot = -0.0881583
                absolute energy error: E_tot - E_init = 0.530692
                relative energy error: (E_tot - E_init) / E_init = -0.857545
time = 10
star1 = 3  star2 = 8  a = 0.00197342  e = 0.806924
|gravity> 
\end{verbatim}
\end{small}

\abc

\bob
What happened?  I had expected it to take three times longer, with a
three times smaller step size criterion; instead it took five times
longer, with more than two and a half million steps instead of the
previous half million.  And what is worse, the energy conservation is
now atrocious!

\alice
The fact that we have embarked on a completely different history is by
itself not surprising.  The $N$-body system is exponentially unstable.
In other words, even a slight chance anywhere in the position or
velocity of any single particle will quickly lead to a completely
different behavior of the system.  This is what is popularly called
the butterfly effect: if a butterfly flaps its wings somewhere in the
world, and if that happens in an exponentially unstable part of a
weather zone, the whole weather pattern on the planet could become
rather different after a few weeks or so from what it otherwise
would have been.  Whether this is really true for actual butterflies,
I don't know, but it is certainly true for stars!

\carol
So we may have just been unlucky, getting not only onto a different
trajectory, but on a trajectory with a bad energy behavior to boot.
But a few days ago, we had much better luck with energy conservation.
How about trying again, but actually degrading the accuracy a bit,
with the option {\st -a 0.02}, in between the default of {\st -a 0.03}
and the {\st -a 0.01} you just tried?  Who knows, we may be more lucky
this time.

\bob
Using a Monte Carlo method?  I didn't think playing roulette was a
very scientific approach!  But what the heck, let's give it a try.

\alice
You may not have heard of it, but there is actually something called a
Monte Carlo method in science!  If we had chosen a rejection technique
to construct our homogeneous spherical particle distribution, we would
have used a type of Monte Carlo technique.

\bob
What do you know.  Well, why don't we call Carol's proposed
hit-and-miss approach a Monte Carol method.

\carol
You'd better start typing instead!

\cba

\begin{small}
\begin{verbatim}
|gravity> ../chap9/sphere1 -s 42 -n 25 | ../chap8/nbody_sh1 -d 0.01 -e 10 -i | find_binaries2 -a 0.1
|gravity> ../chap9/sphere1 -s 42 -n 25 | ../chap8/nbody_sh1 -d 0.02 -e 10 -i | find_binaries2 -a 0.1
seed = 42
Starting a Hermite integration for a 25-body system,
  from time t = 0 with time step control parameter dt_param = 0.02  until time 10 ,
  with diagnostics output interval dt_dia = 10,
  and snapshot output interval dt_out = 1.
at time t = 0 , after 0 steps :
  E_kin = 0 , E_pot = -0.61885 , E_tot = -0.61885
                absolute energy error: E_tot - E_init = 0
                relative energy error: (E_tot - E_init) / E_init = -0
time = 0
star1 = 6  star2 = 12  a = 0.0271531  e = 1
star1 = 8  star2 = 23  a = 0.0783721  e = 1
star1 = 13  star2 = 14  a = 0.0386034  e = 1
time = 1.00047
star1 = 5  star2 = 18  a = 0.0810243  e = 0.423949
star1 = 7  star2 = 14  a = 0.0492761  e = 0.914931
time = 2.00002
star1 = 2  star2 = 3  a = 0.084026  e = 0.496036
star1 = 7  star2 = 14  a = 0.068698  e = 0.706102
star1 = 17  star2 = 19  a = 0.0137171  e = 0.782917
time = 3.00001
star1 = 0  star2 = 19  a = 0.0336442  e = 0.813088
star1 = 4  star2 = 17  a = 0.0157056  e = 0.832995
star1 = 7  star2 = 21  a = 0.0377602  e = 0.835347
star1 = 17  star2 = 18  a = 0.0190899  e = 0.891115
time = 4.00003
star1 = 3  star2 = 10  a = 0.0363436  e = 0.795917
star1 = 5  star2 = 19  a = 0.0861304  e = 0.614849
star1 = 7  star2 = 21  a = 0.0377828  e = 0.83908
star1 = 10  star2 = 17  a = 0.0491587  e = 0.319343
time = 5.00006
star1 = 4  star2 = 18  a = 0.049268  e = 0.70185
star1 = 7  star2 = 21  a = 0.0377876  e = 0.839173
star1 = 8  star2 = 24  a = 0.0812205  e = 0.513454
time = 6.00001
star1 = 4  star2 = 17  a = 0.0142633  e = 0.783939
star1 = 7  star2 = 21  a = 0.0377902  e = 0.838337
star1 = 8  star2 = 24  a = 0.0252418  e = 0.941765
star1 = 17  star2 = 24  a = 0.0727696  e = 0.719213
time = 7.00001
star1 = 4  star2 = 13  a = 0.00581595  e = 0.110655
star1 = 7  star2 = 21  a = 0.0377915  e = 0.836602
time = 8.00002
star1 = 4  star2 = 13  a = 0.00627755  e = 0.258843
star1 = 7  star2 = 21  a = 0.0377902  e = 0.831768
time = 9.00014
star1 = 4  star2 = 23  a = 0.0182906  e = 0.961251
star1 = 7  star2 = 21  a = 0.033901  e = 0.86658
star1 = 13  star2 = 15  a = 0.0571123  e = 0.45941
star1 = 13  star2 = 23  a = 0.0455079  e = 0.890129
at time t = 10 , after 300848 steps :
  E_kin = 0.811683 , E_pot = -0.997429 , E_tot = -0.185746
                absolute energy error: E_tot - E_init = 0.433104
                relative energy error: (E_tot - E_init) / E_init = -0.699853
time = 10
star1 = 4  star2 = 8  a = 0.00255213  e = 0.988263
star1 = 7  star2 = 21  a = 0.0340237  e = 0.880843
star1 = 8  star2 = 24  a = 0.0435369  e = 0.365145
|gravity> 
\end{verbatim}
\end{small}

\abc

\bob
No cigar.  Yes, a completely different run, but no, still a bad energy
behavior.  By now I'm getting pretty curious to see how and when this
bad energy error occurs.  Let's get some more information, once every
unit time increase.

\cba

\begin{small}
\begin{verbatim}
|gravity> ../chap9/sphere1 -s 42 -n 25 | ../chap8/nbody_sh1 -d 0.02 > /dev/null
seed = 42
Starting a Hermite integration for a 25-body system,
  from time t = 0 with time step control parameter dt_param = 0.02  until time 10 ,
  with diagnostics output interval dt_dia = 1,
  and snapshot output interval dt_out = 1.
at time t = 0 , after 0 steps :
  E_kin = 0 , E_pot = -0.61885 , E_tot = -0.61885
                absolute energy error: E_tot - E_init = 0
                relative energy error: (E_tot - E_init) / E_init = -0
at time t = 1.00047 , after 12485 steps :
  E_kin = 1.06276 , E_pot = -1.24842 , E_tot = -0.185664
                absolute energy error: E_tot - E_init = 0.433187
                relative energy error: (E_tot - E_init) / E_init = -0.699986
at time t = 2.00002 , after 21245 steps :
  E_kin = 1.14737 , E_pot = -1.33303 , E_tot = -0.185664
                absolute energy error: E_tot - E_init = 0.433187
                relative energy error: (E_tot - E_init) / E_init = -0.699986
at time t = 3.00001 , after 30349 steps :
  E_kin = 1.20407 , E_pot = -1.38973 , E_tot = -0.185664
                absolute energy error: E_tot - E_init = 0.433186
                relative energy error: (E_tot - E_init) / E_init = -0.699986
^d
|gravity> 
\end{verbatim}
\end{small}

\abc

\carol
That was a good move, Bob!  So the error started right away.  How
strange.  Could you make the energy output interval even shorter?

\bob
I'll make it a hundred times shorter.  See what happens

\cba

\begin{small}
\begin{verbatim}
|gravity> ../chap9/sphere1 -s 42 -n 25 | ../chap8/nbody_sh1 -e 0.01 -d 0.02 > /dev/null
seed = 42
Starting a Hermite integration for a 25-body system,
  from time t = 0 with time step control parameter dt_param = 0.02  until time 10 ,
  with diagnostics output interval dt_dia = 0.01,
  and snapshot output interval dt_out = 1.
at time t = 0 , after 0 steps :
  E_kin = 0 , E_pot = -0.61885 , E_tot = -0.61885
                absolute energy error: E_tot - E_init = 0
                relative energy error: (E_tot - E_init) / E_init = -0
at time t = 0.0106043 , after 12 steps :
  E_kin = 0.00111979 , E_pot = -0.61997 , E_tot = -0.61885
                absolute energy error: E_tot - E_init = 4.737e-10
                relative energy error: (E_tot - E_init) / E_init = -7.65452e-10
at time t = 0.0203781 , after 24 steps :
  E_kin = 0.00454037 , E_pot = -0.62339 , E_tot = -0.61885
                absolute energy error: E_tot - E_init = 6.01715e-10
                relative energy error: (E_tot - E_init) / E_init = -9.72311e-10
at time t = 0.0304865 , after 39 steps :
  E_kin = 0.0123603 , E_pot = -0.63121 , E_tot = -0.61885
                absolute energy error: E_tot - E_init = 7.24435e-10
                relative energy error: (E_tot - E_init) / E_init = -1.17062e-09
at time t = 0.0400792 , after 60 steps :
  E_kin = 0.0317015 , E_pot = -0.650552 , E_tot = -0.61885
                absolute energy error: E_tot - E_init = 3.93875e-11
                relative energy error: (E_tot - E_init) / E_init = -6.36463e-11
at time t = 0.0500072 , after 1790 steps :
  E_kin = 0.399428 , E_pot = -1.13502 , E_tot = -0.735588
                absolute energy error: E_tot - E_init = -0.116738
                relative energy error: (E_tot - E_init) / E_init = 0.188638
at time t = 0.0600297 , after 3663 steps :
  E_kin = 0.55247 , E_pot = -0.738132 , E_tot = -0.185662
                absolute energy error: E_tot - E_init = 0.433188
                relative energy error: (E_tot - E_init) / E_init = -0.699989
|gravity> 
\end{verbatim}
\end{small}

\abc

\alice
Ah, a bad seed, so to speak.  Something dramatic happened already
within the first one tenth of the first time unit!

\carol
It would be interesting to find out more about what happened, and we
do have the tools to do so, armed as we are with movie making equipment
and a binary detector.  We could even write a close encounter detector,
similar to the binary detector but then for unbounded hyperbolic orbits.
Perhaps they were the culprit.

\bob
All nice and fine, but first things first.  Let me just take a
different seed, and finally get a decent 25-body evolution, to see
what our binary detector does under less pathological circumstances.

\cba

\begin{small}
\begin{verbatim}
|gravity> ../chap9/sphere1 -s 43 -n 25 | ../chap8/nbody_sh1 -e 10 -i | find_binaries2 -a 0.1
seed = 43
Starting a Hermite integration for a 25-body system,
  from time t = 0 with time step control parameter dt_param = 0.03  until time 10 ,
  with diagnostics output interval dt_dia = 10,
  and snapshot output interval dt_out = 1.
at time t = 0 , after 0 steps :
  E_kin = 0 , E_pot = -0.569576 , E_tot = -0.569576
                absolute energy error: E_tot - E_init = 0
                relative energy error: (E_tot - E_init) / E_init = -0
time = 0
star1 = 3  star2 = 6  a = 0.0653857  e = 1
star1 = 10  star2 = 23  a = 0.0388408  e = 1
time = 1.00015
star1 = 6  star2 = 13  a = 0.0631629  e = 0.403642
star1 = 10  star2 = 17  a = 0.0668875  e = 0.452269
time = 2.00015
star1 = 6  star2 = 13  a = 0.0413988  e = 0.918359
star1 = 12  star2 = 22  a = 0.0335088  e = 0.255619
star1 = 17  star2 = 18  a = 0.0773984  e = 0.823559
time = 3.00008
star1 = 0  star2 = 22  a = 0.0906416  e = 0.924285
star1 = 4  star2 = 12  a = 0.0261997  e = 0.596176
star1 = 6  star2 = 13  a = 0.0417364  e = 0.86777
star1 = 10  star2 = 16  a = 0.0706447  e = 0.835167
time = 4.0002
star1 = 4  star2 = 22  a = 0.0858038  e = 0.739406
star1 = 4  star2 = 23  a = 0.0933074  e = 0.730645
star1 = 6  star2 = 13  a = 0.0415368  e = 0.860736
time = 5.00032
star1 = 4  star2 = 22  a = 0.0151778  e = 0.983233
star1 = 18  star2 = 23  a = 0.0982301  e = 0.754695
time = 6.00013
star1 = 3  star2 = 10  a = 0.0345493  e = 0.752254
time = 7.00037
star1 = 3  star2 = 4  a = 0.0726365  e = 0.907422
star1 = 3  star2 = 17  a = 0.0738742  e = 0.856752
star1 = 4  star2 = 17  a = 0.0742944  e = 0.94384
time = 8.00001
star1 = 1  star2 = 8  a = 0.0341744  e = 0.990472
star1 = 22  star2 = 23  a = 0.0467334  e = 0.70404
time = 9.00015
star1 = 0  star2 = 4  a = 0.0548707  e = 0.917466
star1 = 3  star2 = 4  a = 0.0186082  e = 0.237707
star1 = 3  star2 = 8  a = 0.0681921  e = 0.993719
at time t = 10.0001 , after 54838 steps :
  E_kin = 0.490667 , E_pot = -1.06032 , E_tot = -0.569648
                absolute energy error: E_tot - E_init = -7.22701e-05
                relative energy error: (E_tot - E_init) / E_init = 0.000126884
time = 10.0001
star1 = 6  star2 = 7  a = 0.0980807  e = 0.998395
star1 = 16  star2 = 22  a = 0.0192647  e = 0.694863
|gravity> 
\end{verbatim}
\end{small}

\abc

\bob
Hehe!  That's a whole lot better.  A relative energy error of about
one hundredth of a percent.  Perhaps a good time to stop $;>)$.

\carol
For now, yes, but I have the feeling that we have just only uncovered
the tip of an iceberg.

\alice
I'm sure of that, too.  I hope we will get back in a while, to delve
further into all this.

\cba

\begin{Exercise}[Close encounters]\label{ex:closeencounters}
Design and build a piece of code similar to {\st find\_binaries2.C},
but now for unbound close encounters between two particles.  Use that
code to analyse in more detail when and how errors become unacceptably
large, as we have seen above.  Can you suggest ways to improve the
error accuracy, other than just setting the overall accuracy parameter
to a smaller and smaller value?
\end{Exercise}
