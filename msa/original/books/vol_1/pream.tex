% preamble

\usepackage{verbatim}\usepackage{epsf}
\usepackage{framed}
\usepackage{html}
\usepackage[dvipdfm]{graphicx}
\usepackage[thmmarks,standard,framed,thref]{ntheorem}

%
% The complete documentation of this ntheorem package is in
% ntheorem.ps
% 
\theoremstyle{break}
\newtheorem{Exercise}{Exercise}[chapter]
\newframedtheorem{Code}{Code}[chapter]
\newframedtheorem{Output}{Output}[chapter]
%\newtheorem{Code}{Code}[chapter]
%\newtheorem{Output}{Output}[chapter]

%
% If you want to add new box, just add a line like
%
%\newframedtheorem{foo}{foo}[chapter]
%
% this will make the foo environment, numbered sequentially within
% chapter, available. So if you say
%
% \begin{foo}
% This is first sample of foo.
% \end{foo}
% You will see this in a framed box
%
% To get a list of this foo box, you need to say something like
%
%\chapter*{List of foos}
%\addcontentsline{toc}{chapter}{List of foos}
%\theoremlisttype{allname}
%\listtheorems{foo}
%
% first two lines are just to make this list looks like list of
% figures and list of tables. 

\setlength{\textwidth}{ 5.45in}           % default seems to be around 4.8 inch,
\addtolength{\evensidemargin}{-0.4in}  % so we can devide the extra 0.65 inch
\addtolength{\oddsidemargin}{-0.4in}   % over left and right margin

\setlength{\parindent}{2.5em}
\setlength{\parskip}{1.2 ex plus0.2ex minus 0.1ex}
\renewcommand{\baselinestretch}{1.05}
\renewcommand{\floatpagefraction}{1.0}
\renewcommand{\topfraction}{1.0}

\begin{htmlonly}
\setlength{\textheight}{30in}
\end{htmlonly}
