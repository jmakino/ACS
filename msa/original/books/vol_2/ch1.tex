\chapter{The Two-Body Problem}

{\bf Here I will collect equations with some brief text, as
introductory material -- Piet}

\section{equation of motion}

To start with, let us recall the equations of section 2.3 in volume 1.

Newton's gravitational equation of motion for the vector $\br = \br_2 - \br_1$
pointing from particle 1 to particle 2:

\begin{equation}
\frac{d^2}{dt^2}\br = - G \frac{M_1 + M_2}{r^3}\br
\end{equation}

Here $r = | \br |$ is the absolute value of the separation vector $\br$,
the masses of the two particles are indicated by $M_1$ and $M_2$,
respectively, and $G$ is the value of Newton's gravitational constant.


Let us again choose units such that

\begin{eqnarray}
G & = & 1 \\
M_1 + M_2 & = & 1
\end{eqnarray}

which leads to the simplified equation of motion

\begin{equation}
\frac{d^2}{dt^2}\br = - \frac{\br}{r^3} \label{newton-2-bodies}
\end{equation}

\section{1D: time transformation only}

Let us start with the one-dimensional case, to see how we can deal
with the singularity in the equation:

\begin{equation}
\frac{d^2r}{dt^2} = - \frac{1}{r^2}
\end{equation}

We will use the shorthand notation of a dot above a quantity to denote
the time derivative, \ie $\dot{\alpha} = d\alpha/dt$.  This
allows us to write:

\begin{equation}
\ddot{r} = - \frac{1}{r^2} \label{eom}
\end{equation}

Note that the energy

\begin{equation}
h = \dhalf\dot{r}^2 - \frac{1}{r} \label{energy}
\end{equation}

is conserved:

\begin{equation}
\dot h = \dot r \ddot r + \frac{\dot r}{r^2} =
         - \frac{\dot r}{r^2} + \frac{\dot r}{r^2} = 0
\end{equation}

Let us try to get rid of the singularity in the equation of motion \ref{eom}
through a time transformation, introducing a variable $\tau$
determined by the differential equation

\begin{equation}
dt = r^\alpha d\tau
\end{equation}

Let us indicate differentiation with respect to $\tau$ by a prime,
\ie $\alpha' = d\alpha/d\tau$.

We have:

\begin{equation}
r' = \frac{dr}{d\tau} = \frac{dt}{d\tau} \frac{dr}{dt} = r^\alpha \frac{dr}{dt}
\end{equation}

and:

\begin{eqnarray}
r''=\frac{d^2r}{d\tau^2} &=& \alpha r^{\alpha-1}\frac{dr}{d\tau}\frac{dr}{dt}
     + \frac{dt}{d\tau} r^\alpha \frac{d^2r}{dt^2}
= \alpha r^{2\alpha-1}\dot r^2 + r^{2\alpha} \ddot r \nonumber \\
&=& \alpha r^{2\alpha-1}\left(2h+\frac{2}{r}\right)
 + r^{2\alpha} \left(- \frac{1}{r^2}\right) 
\end{eqnarray}

or

\begin{eqnarray}
r'' = 2h\alpha r^{2\alpha-1} + (2\alpha-1)r^{2\alpha-2}
\end{eqnarray}

There are two natural choices for $\alpha$ that present themselves.

\begin{eqnarray}
\fbox{$\alpha=1$:}&\qquad& r''=2hr+1 \qquad\qquad\qquad\qquad\qquad\qquad \\
{\rm and}\qquad\qquad\qquad\qquad\qquad\qquad&\qquad& \nonumber \\
\fbox{$\alpha=\dhalf$:}&\qquad& r''=h \qquad\qquad\qquad\qquad\qquad\qquad
\end{eqnarray}

The choice $\alpha=1$ leads to an equation of motion for a displaced
harmonic oscilator.  The choice $\alpha=\dhalf$ leads to an even
simpler equation of motion, similar to that for an object falling in a
constant gravitational field.  If we choose the initial condition $r(0)=0$,
the general solution is:

\begin{equation}
\fbox{$\alpha=\dhalf$:}\qquad r(\tau) = A\tau + \dhalf h \tau^2
\end{equation}

Note that for $\tau \rightarrow 0$ (and hence $r \rightarrow 0$ and
$t \rightarrow 0$), Eq. \ref{energy} tells us that we may as well
take $h=0$ to first order, since the other two terms grow without bound.
This leads to the approximation:

\begin{equation}
A=r'=r^\dhalf \dot r = r^\dhalf \left( \frac{2}{r} \right)^\dhalf = \sqrt{2}
\end{equation}

which leads to the full solution:

\begin{equation}
\fbox{$\alpha=\dhalf$:}\qquad r(\tau) = \sqrt{2}\tau + \dhalf h \tau^2
\end{equation}





\section{xxx}

\section{yyy}

\section{2D: time transformation only}

Douglas: I think the time transformation in 2d works OK, as follows:

\begin{eqnarray}
{d\br\over dt} &=& {1\over r}{d\br\over d\tau}, \\
{d^2\br\over dt^2} &=&
-{1\over r^2}\dot r{d\br\over d\tau} + {1\over r^2}{d^2\br\over d\tau^2}, \\
&=& -{\br\over r^3}
\end{eqnarray}

and so

\begin{eqnarray}
{d^2\br\over d\tau^2} &=& -{\br\over r} + r\dot r\dot\br, \\
&=& 2h\br + r\dot r\dot\br - \dot\br^2\br + {\br\over r}, \\
&=& 2h\br - \be,
\end{eqnarray}

where the Lenz vector is 

\begin{eqnarray}
\be &=&  - r\dot r\dot\br + \dot\br^2\br - {\br\over r}, \\
&=& \dot\br\times(\br\times\dot\br) - {\br\over r}
\end{eqnarray}

which fortunately is the same result as in Chapter 15!
