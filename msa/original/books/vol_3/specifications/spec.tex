\documentclass{article}[12pt]
\usepackage{verbatim}\usepackage{epsf}
\usepackage{subsubsection}
\setcounter{secnumdepth}{7}
\setcounter{tocdepth}{7}

\setlength{\textwidth}{16cm}            % default seems to be around 4.8 inch,
\addtolength{\evensidemargin}{-0.2in}    % so we can devide the extra 0.4 inch
\addtolength{\oddsidemargin}{-0.2in}     % equally over left and right margin

\setlength{\parindent}{2.5em}
\setlength{\parskip}{1.2 ex plus0.2ex minus 0.1ex}
\renewcommand{\baselinestretch}{1.05}
\renewcommand{\floatpagefraction}{1.0}
\renewcommand{\topfraction}{1.0}

\def\filler{xxxx xxxx xxxx xxxx xxxx xxxx xxxx xxxx xxxx xxxx xxxx xxxx
xxxx xxxx xxxx xxxx xxxx xxxx xxxx xxxx xxxx xxxx xxxx xxxx
xxxx xxxx xxxx xxxx xxxx xxxx xxxx xxxx xxxx xxxx xxxx xxxx
xxxx xxxx xxxx xxxx xxxx xxxx xxxx xxxx xxxx xxxx xxxx xxxx
xxxx xxxx xxxx xxxx xxxx xxxx xxxx xxxx xxxx xxxx xxxx xxxx
xxxx xxxx xxxx xxxx xxxx xxxx xxxx xxxx xxxx xxxx xxxx xxxx
xxxx xxxx xxxx xxxx xxxx xxxx xxxx xxxx xxxx xxxx xxxx xxxx
xxxx xxxx xxxx xxxx xxxx xxxx xxxx xxxx xxxx xxxx xxxx xxxx}
\def\filler{}

\begin{document}

\title{Kali: A Gravitational Many-Body Code}

\author{Piet Hut \& Jun Makino}

\maketitle

\begin{abstract}

xxx

yyy

xxx

\end{abstract}

\newpage

\tableofcontents

\newpage

\section{Introduction}

\filler

\subsection{A Brief History of Many-Body Codes}

\filler

\subsubsection{Prehistory}

Hand Calculations by Chazy and Copenhagen groups

Using photoelectric cells

First calculations in Los Alamos(?)

\subsubsection{Diversification}

Cambrian explosion in the sixties.

In the seventies and eighties: Aarseth's codes the only game in town
for collisional stellar dynamics, or as it became called later: the
dynamics of dense stellar systems.

In the nineties: the Kira code, as the only competitor.

\subsubsection{NBODY4}

Mention the various NBODYx, and the prominent role of NBODY4, and its
main structure.  Pros and cons.

\subsubsection{Kira}

Main structure.  Pros and cons.

\subsection{Modeling Dense Stellar Systems}

\filler

\subsubsection{Stellar Evolution and Binary Stellar Evolution}

The need for stellar evolution.

From stellar dynamics: need to treat physical encounters within dynamics.

From stellar evolution: need to make population studies dynamic.

The problem that stellar evolution codes don't run by themselves.

\subsubsection{Hydrodynamics of Stellar Collisions}

From hydrodynamics: need cluster models to generate relevant collisions.

Collision conference 2000 at AMNH.

\subsubsection{Stellar Dynamics in Various Approximations}

Mention Monte Carlo codes, Fokker-Planck, Gas Codes, etc.

\subsubsection{Recipe-based Stellar Evolution: BSE and SeBa}

\filler

\subsubsection{MODEST workshops, reviews, working groups}

Brief history.  Main goals.  Current situation (needs frequent updates).

\subsection{A New Approach}

\filler

\subsubsection{Thorny Problems in Existing Many-Body Codes}

\filler

\subsubsection{The Need for a Protocol to Connect Codes}

\filler

\subsubsection{The Kali Code: A Top-Down Approach to Specification}

Kali is the name of an Indian Goddess, associated with time ({\it
kala} is the Sanskrit word for time).  Kali is associated with wild
motion and destruction but also with freedom from fear.  She is the
embodiment of the energy and power of time and manifest in both benign
and terrible ways.  Kali is black, symbolizing the Night of Time; She
straddles the cosmic eons.  Her four arms represent the four directons
of space identified with the complete cycle of Time.  The name of the
Indian city Calcutta is actually an anglicized version of {\it
kali-ghatt}, or `steps of kali', Calcutta being Kali's temple city.

\clearpage
\newpage

\section{A Modular Approach}

\filler

\subsection{The Need for Flexibility and Robustness}

\filler

\subsubsection{Stellar Evolution Codes as Black Boxes}

\filler

\subsubsection{Mixing Various Languages and Implementations}

\filler

\subsubsection{The Use of Special-Purpose Hardware}

\filler

\subsubsection{Anticipating Unforseen Developments}

\filler

\subsection{The Role of Specifications}

\filler

\subsubsection{Language Independent Descriptions}

\filler

\subsubsection{Validation}

\filler

\subsection{The Importance of Interface Definitions}

\filler

\subsubsection{Interfaces on Many Levels}

\filler

\subsubsection{Simplicity and Precision}

\filler

\clearpage
\newpage

\section{Top-Level Structure}

\filler

\subsection{Stellar Dynamics, Hydrodynamics and Stellar Evolution}

\filler

\subsection{The Global-Local Structure of Stellar Dynamics}

\begin{figure}[htb]
\begin{center}
\epsfxsize = 3in
\epsffile{top1.eps}
\caption
{Top-level view xxx xxx xxx xxx}
\label{fig:top1}
\end{center}
\end{figure}

%\begin{figure}[htb]
%\begin{center}
%\epsfxsize = 3.5in
%\epsffile{top2.eps}
%\caption
%{Top-level view xxx xxx xxx xxx}
%\label{fig:top2}
%\end{center}
%\end{figure}

\filler

\subsubsection{Global Stellar Dynamics}

\filler

\subsubsection{Local Stellar Dynamics}

\filler

\subsubsection{Stellar Physics}

\filler

\subsubsection{Master Scheduler}

\begin{figure}[htb]
\begin{center}
\epsfxsize = 3.5in
\epsffile{top3.eps}
\caption
{Top-level view xxx xxx xxx xxx}
\label{fig:top3}
\end{center}
\end{figure}

\filler

\subsection{Connectivity between Modules}

\filler

\subsubsection{Global-Local Conversion}

\filler

\subsubsection{Local-Stellar Conversion}

\filler

\subsubsection{Global-Local-Stellar Coupling}

\filler

\subsubsection{Master-[Global/Local/Stellar] Sequencing}

\filler

\clearpage
\newpage

\section{Global Stellar Dynamics}

\filler

\subsection{Overview}

\begin{figure}[htb]
\begin{center}
\epsfxsize = 2.5in
\epsffile{gsd1.eps}
\caption
{Global Dynamics overview xxx xxx xxx xxx}
\label{fig:gsd1}
\end{center}
\end{figure}

\begin{figure}[htb]
\begin{center}
\epsfxsize = 3.5in
\epsffile{gsd2.eps}
\caption
{Global Dynamics detailed overview xxx xxx xxx xxx}
\label{fig:gsd2}
\end{center}
\end{figure}

\clearpage

\filler

\subsubsection{Internal Structures}

\filler

\subsubsection{External Interfaces}

\filler

\subsection{Scheduler Module}

\filler

\subsubsection{Structure}

\filler

\subsubsection{Interfaces}

\filler

\subsubsubsection{Interface with Orbit Module}

\filler

\subsubsubsection{Global-Master Sequencing Interface}

\filler

\subsection{Orbit Module}

\filler

\subsubsection{Structure}

\filler

\subsubsection{Interfaces}

\filler

\subsubsubsection{Interface with Scheduler Module}

\filler

\subsubsubsection{Interface with Interaction Module}

\filler

\subsubsubsection{Global-Local Conversion Interface}

\filler

\subsubsubsection{Global-Stellar Coupling Interface}

For changes in mass and radius.

\subsection{Interaction Module}

\filler

\subsubsection{Structure}

\filler

\subsubsection{Interfaces}

\filler

\subsubsubsection{Interface with Orbit Module}

\filler

\subsubsubsection{Global-Local Coupling Interface}

\filler

\subsubsubsubsection{Global-to-Local Coupling Interface}

\filler

\subsubsubsubsection{Local-to-Global Coupling Interface}

\filler

\subsubsubsection{Global-Stellar Coupling Interface}

For example, to take into account the effects of the quadrupole
moments of isolated contact binaries.

\subsubsubsection{Global-to-Stellar Coupling Interface}

\filler

\subsubsubsection{Stellar-to-Global Coupling Interface}

If a single star has a large flattening, and therefore a quadrupole
moment that can influence nearby stars, then that star will be flagged
through the `f' factor in $r_{ij} < f(R_i + R_J)$ as neading a `hydro'
type treatment (which could be just a simple formula).

All contact binaries always send three pseudoparticles to the global
module to communicate their quadrupole moment.

\clearpage
\newpage

\section{Local Stellar Dynamics}

\filler

\subsection{Overview}

\begin{figure}[htb]
\begin{center}
\epsfxsize = 3.5in
\epsffile{lsd1.eps}
\caption
{Local Dynamics overview xxx xxx xxx xxx}
\label{fig:lsd1}
\end{center}
\end{figure}

\begin{figure}[htb]
\begin{center}
\epsfxsize = 4.5in
\epsffile{lsd2.eps}
\caption
{Local Dynamics detailed overview xxx xxx xxx xxx}
\label{fig:lsd2}
\end{center}
\end{figure}

\clearpage

\filler

\subsubsection{Internal Structures}

\filler

\subsubsection{External Interfaces}

\filler

\subsection{Scheduler Module}

\filler

\subsubsection{Structure}

\filler

\subsubsection{Interfaces}

\filler

\subsubsubsection{Interface with Orbit Module}

\filler

\subsubsubsection{Local-Master Sequencing Interface}

\filler

\subsection{Orbit Module}

\filler

\subsubsection{Structure}

\noindent
For `collapsed' binaries:

1) if the outside perturbing tidal force is more than $10^{-3}$ times the
internal forces, the perturbation is considered strong.

2) if the perturber is closer than 10 times the radius of the binary, the
perturbation is also considered strong, even if it does not influence
the binary very much (if the perturber is very light, for example), since
the binary is likely to influence the other particle significantly.

\noindent
For `quarantined' multiples, we need a similar criterion, perhaps with
different coefficients.

\subsubsection{Interfaces}

\filler

\subsubsubsection{Interface with Scheduler Module}

Warn about occurrence of slingshot events producing new high-velocity stars.

\subsubsubsection{Interface with Interaction Module}

\filler

\subsubsubsection{Interface with Fusion Module}

\filler

\subsubsubsection{Interface with Fission Module}

\filler

\subsubsubsection{Local-Stellar Coupling Interface}

\filler

\subsection{Interaction Module}

\filler

\subsubsection{Structure}

\filler

\subsubsection{Interfaces}

\filler

\subsubsubsection{Interface with Orbit Module}

\filler

\subsubsubsection{Local-Global Coupling Interface}

\filler

\subsubsubsection{Local-Lobal Coupling Interface}

\filler

\subsubsubsection{Local-Stellar Coupling Interface}

\filler

\subsection{Fusion Module}

\filler

\subsubsection{Structure}

\filler

\subsubsection{Interfaces}

\filler

\subsubsubsection{Interface with Orbit Module}

\filler

\subsubsubsection{Local-Global Fusion Conversion Interface}

\filler

\subsection{Fission Module}

\filler

\subsubsection{Structure}

\filler

\subsubsection{Interfaces}

\filler

\subsubsubsection{Interface with Orbit Module}

\filler

\subsubsubsection{Local-Global Fission Conversion Interface}

\filler

\clearpage
\newpage

\section{Stellar Physics}

\filler

\subsection{Overview}

\begin{figure}[htb]
\begin{center}
\epsfxsize = 2.5in
\epsffile{sp1.eps}
\caption
{Stellar Physics overview xxx xxx xxx xxx}
\label{fig:sp1}
\end{center}
\end{figure}

\begin{figure}[htb]
\begin{center}
\epsfxsize = 3.5in
\epsffile{sp2.eps}
\caption
{Stellar Physics detailed overview xxx xxx xxx xxx}
\label{fig:sp2}
\end{center}
\end{figure}

\clearpage

\filler

\subsection{}

\filler

\subsection{}

\filler

\subsection{}

\filler

\clearpage
\newpage

\section{Conclusion}

\filler

\subsection{}

\filler

\subsection{}

\filler

\end{document}
