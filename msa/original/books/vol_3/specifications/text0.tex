\documentclass{article}[12pt]
\usepackage{verbatim}\usepackage{epsf}

\setlength{\textwidth}{16cm}            % default seems to be around 4.8 inch,
\addtolength{\evensidemargin}{-0.2in}    % so we can devide the extra 0.4 inch
\addtolength{\oddsidemargin}{-0.2in}     % equally over left and right margin

\setlength{\parindent}{2.5em}
\setlength{\parskip}{1.2 ex plus0.2ex minus 0.1ex}
\renewcommand{\baselinestretch}{1.05}
\renewcommand{\floatpagefraction}{1.0}
\renewcommand{\topfraction}{1.0}

\begin{document}
\begin{itemize}

\item for equal mass particles, two particles can be expected to
   stay in a Kepler orbit reasonably long (and thus can be
   analytically Kepler regularized) is there separation
   $r_{ij}$ obeys $r_{ij} < r_K$ with $r_K \propto N^{-3/7}$.
   This follows from the requirement that the Kepler orbital
   time $T < \tau$ where $\tau = 1/(n \sigma v)$ is the
   encounter time, with $\sigma$ the geometrical cross section
   and $v$ the typical velocity.  Then $\tau = 1/(N.r_K^2.1)$
   and $T=(N.r_K^3)^{-1/2}$, hence $r_K \propto N^{-3/7}$.

\item for the Global-Local fusion triggering, the global interaction
   module flags pairs of particles as candidates for merging, but
   then the orbit module only sends two particles at a time to
   the local module.  If one particle wants to consider two other
   particles as candidates to be merged with, the orbit module
   decides, perhaps just randomly, which of those candidates to
   offer to the local module first.  Here is the scheduling:
\begin{itemize}
   \item do a block time step, which returns k particle pair candidates
   \item send all nonoverlapping particle pairs to the local module
   \item if none are accepted, send one or more remaining pairs
   \item whenever a change is accepted, rerun the force calculator
   \item if nothing new is accepted, and no pairs remain, take the
      the next block time step.
\end{itemize}

\item each atom or molecule, whether it is a gravitational collection
   of mass points within the local module, or an SPH region within
   the stellar physics module, or a contact binary or single
   star (with possibly tidal bulges), has a radius $R$ that is
   defined exactly as the largest distance from the center of
   mass of that atom or molecule of any bit of mass that is still
   dynamically significant. The $f$ factor that determines whether
   two (atom/molecule)s are candidates for merging is hardwired
   to be a certain factor $f_{local}$, typically $1.2$ or $1.3$
   or so, for conversion from the global to the local module.
   For the conversion from the local to the stellar-physics module,
   the $f$ factor is provided by the stellar-physics module, and
   it is typically 3 to 10.  This same $f$ value will be used
   whenever we consider the fusion of two stellar-physics objects,
   independent of whether each object is a single star or a
   contact binary or an SPH region.

\item conceptually, the data tree of a star cluster is like what
   we use in kira, where each kira leaf is a node with $10^{60}$
   or more daughters, one for each nucleus or electron or photon
   inside a star.  The top node has somewhat less than $N$
   daughters, and the next level nodes are either single stars
   or double stars or SPH regions, or they are gravitational
   molecules, each of which has up to ten or so daughters in
   the nearly equal mass case, or up to $10^4$ to $10^5$ daughters
   in case of a very massive black hole.  A central point: we will
   never do a complete treewalk as is done in kira, since we now
   have strict abstraction boundaries between horizontal levels
   in this tree; the implementer of one level does not need to
   know how a deeper level is handled.

\item how did we derive the loss-of-digits criterion of
   $r_{ij} = \epsilon_{trigger}\sqrt{m_i+m_j}$, with 
   $\epsilon_{trigger} \sim 10^{-2}$ ?
        We start with the requirement that we want to protect
   the energy conservation of the whole system, to within a small
   error tolerance of, say, $10^{-11}$ in N-body units.
        Note that by doing so, we sacrifice accuracy for
   very very light particles.  For example, Earth and Moon
   in a globular cluster of $10^6$ stars, have a total mass
   of $10^{-12}$, so the criterion reads $r_{ij} = 10^{-8}$,
   so you are reduced to single precision integration,
   effectively.  Interestingly, $r_{ij} = 10^{-8}$ corresponds
   to a distance ten times larger than the distance to the moon;
   or anything within the current Hills sphere of the Earth is
   regularized. The round-off distance, by the way, is a couple
   hundred meters.
        Anyway, the absolute energy error is $\epsilon \sim
   \frac{m_j}{r_{ij}}\frac{r_{roundoff}}{r_{ij}} < \epsilon_{safe}$.
   This implies $r_{ij}^2>m_j \frac{r_{roundoff}}{\epsilon_{safe}}$
   or for the trigger radius to start fusion: $r_{trigger} 
   = \sqrt{\frac{r_{roundoff}}{\epsilon_{safe}}}\sqrt{m_i+m_j}$
   With $r_{roundoff} \sim 10^{-15}$ and $\epsilon_{safe}=10^{-11}$
   we have $r_{trigger} = 10^{-2}\sqrt{m_i+m_j}$.

\item how did we derive the Kepler integration accuracy criterion of
   $r_{ij} = 10^{-1}\sqrt{m_i+m_j}$ ?  We start with the value
   of $10^{-2}$ for the coefficient for a single close encounter.
   Assuming that a Kepler orbit gets mostly regularized through
   analytical Kepler integration, we have perhaps $10^4$ orbits
   that need to be numerically integrated.  If the error adds
   in square root fashion, it will grow only by a factor of
   $10^{-2}$.  As we saw above, $\epsilon_{trigger}$ grows only
   with the square root of the energy error, so it becomes only
   $10$ times larger than the previous value of $10^{-2}$.

\item strong perturbations from an external object on a molecule
   would not necessarily cause that object to become part of
   the molecule.  For example, a very heavy black hole with not
   cause everything nearby to become part of one molecule unless
   a loss-of-digit criterion or other criterion is triggered.
\end{itemize}

\clearpage
\large
\begin{minipage}[b]{7.9cm}

\noindent
Candidate:
\begin{enumerate}
\item $ r_{ij} < f(R_i+R_j)$
\item $ r_{ij} < f{\rm max}[R_i(m_j/m_i)^{1/3},R_j(m_i/m_j)^{1/3}]$
\item $ r_{ij} < R_i^{mol}+R_j^{mol})$
\item $ r_{ij} < \epsilon_{trig}\sqrt{m_i+m_j}$, $\epsilon_{trig} \sim 10^{-2}$
\item $ r_{ij} < \epsilon_{trig}^{kep}\sqrt{m_i+m_j}$ {\tt \&\&
bound},

$\epsilon_{trig}^{kep} \sim 10^{-1}$
\end{enumerate}

\end{minipage}
\begin{minipage}[b]{7.9cm}
Candidate:
\begin{enumerate}
\item $ r_{ij} < f(R_i+R_j)$
\item $g(m_{1},r_{1},v_{1},R_{1}, m_{2},r_{2},v_{2},R_{2})$

\end{enumerate}
\end{minipage}


\epsfxsize \columnwidth
\epsffile{top_conversion.eps}

\begin{center}
check
\begin{eqnarray}
1\le f \le 10 & {\rm okay}\nonumber\\
 f > 10 & {\rm warning}\nonumber\\
 f < 1 & {\rm error}\nonumber
\end{eqnarray}
\end{center}
\end{document}