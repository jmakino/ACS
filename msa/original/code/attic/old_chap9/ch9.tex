\chapter{Generating Initial Conditions}

\section{A few Simple Tools}

\code{chap9/pmk_circ_binary.C}

\code{chap9/mk_fig8_triple.C}

\code{chap9/mk_binary_binary.C}

\section{Using a Makefile}

\code{chap9/Makefile}

\section{Setting up a Plummer Model}

\code{chap9/plummer.C}



%% From old version of chapter 7:
%%
%%\section{A More Modular Leapfrog}
%%
%%After having drawn up the first wish list above, Alice, Bob, and Carol
%%sat down to rewrite their leapfrog code to comply with these new
%%wishes.  Although they were in close contact during their code
%%writing, they wound up with three versions, differing only in the way
%%the internal data were stored.  As for the external data format, they
%%all agreed upon the same convention, which meant that their programs
%%could share data from other versions, using the output from those
%%versions as input for their own program, and conversely.  They also
%%stuck to the same logical flow in their three programs.
%%
%%We will first have a look at the program that Bob wrote, which is the
%%simplest of the three, in that it does not use any STL additions to
%%the C++ language.  Those who are familiar with C, the computer
%%language on which C++ was originally modeled, will recognize that
%%Bob's program {\st nbody\_leap1.C} is written pretty much in a C style
%%fashion.  Following that, we will see how Alice introduced the notion
%%of a {\sl vector} from the Standard Template Library to store the
%%particles in an $N$-body system.  Finally, we show how Carol extended
%%Alice's use of vectors to cover the internal data representation for a
%%single particle.
%%
%%Here is Bob's code:
%%
%%\code{chap7/nbody_leap1.C}
%%
%%\section{Comments}
%%
%%\section{Functions and their Arguments}
%%
%%\section{Command Line Options}
%%
%%\section{I/O Control and External Data Structures}
%%
%%\section{Internal Data Structures}
%%
%%\section{Diagnostics}
%%
%%\section{The Driver for the Integrator}
%%
%%\section{The Leapfrog Integrator}
%%
%%\section{The Core Loop: Where the Action is}
%%
